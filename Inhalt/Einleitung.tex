\chapter{Einleitung}
\label{cha:Einleitung}

\section{Motivation und Anwendungsszenario}
In Deutschland ist seit dem Erfolg der Nationalmannschaft bei der Basketball-Weltmeisterschaft 2023 eine deutliche Reichweitensteigerung in der Sportberichterstattung 
zu beobachten~\cite{sinus2025mehralsmedaillen}. Das WM-Finale 2023 verfolgten im ZDF im Durchschnitt rund 4,63 Mio. Zuschauerinnen und Zuschauer~\cite{weis2023topquoten}. Eine solche öffentliche Präsenz kann die Nachfrage 
nach niedrigschwelligen Einstiegsmöglichkeiten und trainingsbegleitenden Angeboten im Basketball begünstigen. Für frühe, reproduzierbare Erfolgserlebnisse ist dabei 
insbesondere die Qualität der Wurfausführung entscheidend. Daraus ergibt sich ein praxisrelevantes Ziel: Trainingsprozesse durch datenbasierte, technologische 
Hilfsmittel so zu unterstützen, dass Wurfleistungen gezielt und nachvollziehbar verbessert werden können. Ein zentraler Ansatzpunkt ist die quantitative Analyse von Ballflugbahnen. 

Basketballwürfe sind das Ergebnis hochkomplexer biomechanischer und physikalischer Abläufe. Bereits geringe Abweichungen in 
Abwurfwinkel, Abwurfgeschwindigkeit oder Abwurfhöhe können die Flugbahn und damit den Wurferfolg maßgeblich beeinflussen. Für eine methodisch kontrollierbare 
Untersuchung bietet sich der Freiwurf an. Er wird aus einer definierten Position ausgeführt, unter weitgehend konstanten Randbedingungen und ohne unmittelbaren 
defensiven Druck. Damit reduziert sich die Variabilität externer Störeinflüsse, was den Freiwurf als standardisierte Basissituation für Modellierung und Vergleich 
besonders geeignet macht.

Auch aus sportpraktischer Perspektive ist der Freiwurf relevant. Empirische Analysen aus dem US-amerikanischen College-Basketball zeigen, dass rund 20\,\% aller erzielten 
Punkte durch Freiwürfe resultierten. Ergänzend wird berichtet, dass bei gleichstarken Teams in 80\,\% der Fälle jene Mannschaft gewinnt, die über die höhere Freiwurfquote 
verfügte ~\cite{kozar1994}. Der Freiwurf ist damit nicht nur ein grundlegendes technisches Element, sondern auch ein leistungsentscheidender Faktor. 
Folglich eignet er sich als Untersuchungseinheit dieser Arbeit, um Wurfdaten unter kontrollierten Bedingungen systematisch auszuwerten.

Vor diesem Hintergrund eröffnet Maschinelles Lernen einen technologischen Rahmen, um aus Wurfdaten Trajektorien vorherzusagen und Rückschlüsse auf zugrunde liegende 
Abwurfparameter zu ziehen. Diese Arbeit grenzt sich dabei bewusst von der Fragestellung ab, eine allgemeingültige „optimale“ Flugbahn herzuleiten, 
die Sportlerinnen und Sportler idealerweise reproduzieren sollten. Im Mittelpunkt steht stattdessen ein personenbezogenes Anwendungsszenario im Sinne 
einer personalisierten Trainingsassistenz: Auf Basis der Daten eines spezifischen Sportlers bzw. einer spezifischen Sportlerin soll ein neuronales Netz individuelle Muster 
identifizieren und daraus sportartspezifisch interpretierbare Hinweise für die Verbesserung ableiten. Der praktische Mehrwert liegt damit nicht in einer normativen 
Vorgabe, sondern in diagnostischer Unterstützung. Aus den Analysen sollen konkrete, sportartspezifisch interpretierbare Hinweise entstehen.

\section{Zielsetzung und Forschungsfragen}
Diese Arbeit zielt auf eine datenbasierte Analyse von Basketball-Freiwürfen unter kontrollierten Bedingungen ab. Da kein öffentlich verfügbarer Datensatz zur Verfügung 
steht, wird zunächst ein synthetischer Datensatz erstellt, der 2D-Ballflugbahnen aus einem physikalisch motivierten Bewegungsmodell ableitet. Hierzu wird ein 
ballistisches Modell implementiert, das zentrale Einflussgrößen des Freiwurfs abbildet und Trajektorien als Funktion relevanter Abwurfparameter generiert. 
Der Datensatz dient anschließend als definierte Grundlage zur Entwicklung, zum Vergleich und zur Bewertung lernbasierter Verfahren.

Auf dieser Basis werden zwei neuronale Modellansätze konzipiert und experimentell gegenübergestellt. Der erste Ansatz verfolgt eine rein datengetriebene Modellierung, 
während der zweite Ansatz physikalisches Vorwissen explizit in die Modellierung einbindet, um physikkonsistente Vorhersagen zu begünstigen. Der Vergleich erfolgt 
entlang dreier Aufgabentypen, die unterschiedliche Aspekte des Problems abdecken:
\begin{itemize}
    \item \textbf{(1):}Trajektorienvorhersage aus variierenden Eingaberepräsentationen
    \item \textbf{(2):}Parameterrekonstruktion als inverse Modellierung 
    \item \textbf{(3):}Wurfklassifikation der vorhergesagten Trajektorien in die Kategorien „zu kurz“, „Treffer“ und „zu lang“
\end{itemize}
Zur Einordnung der Praxistauglichkeit wird das leistungsstärkste Modell anschließend einer Robustheitsanalyse unterzogen, insbesondere gegenüber Variation der 
Eingabequalität und Störeinflüssen. Abschließend wird diskutiert, in welchem Umfang sich aus den Modellausgaben interpretierbare Hinweise für eine diagnostische 
Trainingsassistenz ableiten lassen. Aus diesen Zielen ergeben sich die folgenden Forschungsfragen:
\begin{itemize}
    \item \textbf{F1:} In welchem Umfang erlauben synthetisch generierte, physikalisch motivierte Trajektoriendaten eine valide Evaluation von Lernverfahren zur Freiwurf-Analyse?
    \item \textbf{F2:} Wie wirkt sich die explizite Einbindung physikalischen Vorwissens auf die Leistung in Trajektorienvorhersage, Parameterrekonstruktion und Wurfklassifikation aus?
    \item \textbf{F3:} Wie sensitiv reagieren die Modelle auf die Qualität und den Informationsgehalt der Eingabedaten, und welchen Einfluss hat dies auf die Vorhersagegenauigkeit?
    \item \textbf{F4:} Welche modellbasierten Kenngrößen eignen sich, um aus den Analysen nachvollziehbare, sportartspezifisch interpretierbare Empfehlungen im Sinne einer diagnostischen Trainingsassistenz abzuleiten?
\end{itemize}


\section{Aufbau der Arbeit}

Kapitel~\ref{cha:grundlagen} legt die theoretischen Grundlagen dar und führt die für die Arbeit zentralen Konzepte ein, die zur Einordnung und Interpretation der 
Ergebnisse erforderlich sind. Daran anschließend erfolgt in Kapitel~\ref{cha:verwandteArbeiten} eine Einordnung der Arbeit in den aktuellen Forschungsstand sowie 
eine Erörterung verwandter Ansätze zur Trajektorienvorhersage und physikinformierter Verfahren im Sport.

Die konkrete Aufgabenstellung wird in Kapitel~\ref{cha:problemstellung} präzisiert; dabei werden die zu lösenden Teilprobleme strukturiert und die Zielkriterien für 
die Evaluation abgeleitet. Kapitel~\ref{cha:methodik_und_modelle} beschreibt darauf aufbauend die Methodik, einschließlich Datengenerierung, Modellarchitekturen 
und Versuchsplan zur vergleichenden Bewertung.

Kapitel~\ref{cha:ergebnisse_und_diskussion} präsentiert die empirischen Resultate und diskutiert sie systematisch im Bezug auf die Forschungsfragen. 
Abschließend fasst Kapitel~\ref{cha:fazit} die wesentlichen Erkenntnisse zusammen, reflektiert ihre Aussagekraft und skizziert Perspektiven für weiterführende 
Untersuchungen.
