\chapter{Einleitung}
\label{cha:Einleitung}

\section{Motivation und Anwendungsszenario}
In Deutschland ist seit dem Erfolg der Nationalmannschaft bei der Basketball-Weltmeisterschaft 2023 eine deutliche Reichweitensteigerung in der Sportberichterstattung 
zu beobachten[1]. Das WM-Finale 2023 verfolgten im ZDF im Durchschnitt rund 4,63 Mio. Zuschauerinnen und Zuschauer[2]. Eine solche öffentliche Präsenz kann die Nachfrage 
nach niedrigschwelligen Einstiegsmöglichkeiten und trainingsbegleitenden Angeboten im Basketball begünstigen. Für frühe, reproduzierbare Erfolgserlebnisse ist dabei 
insbesondere die Qualität der Wurfausführung entscheidend. Daraus ergibt sich ein praxisrelevantes Ziel: Trainingsprozesse durch datenbasierte, technologische 
Hilfsmittel so zu unterstützen, dass Wurfleistungen gezielt und nachvollziehbar verbessert werden können. Ein zentraler Ansatzpunkt ist die quantitative Analyse von Ballflugbahnen. 

Basketballwürfe sind das Ergebnis hochkomplexer biomechanischer und physikalischer Abläufe. Bereits geringe Abweichungen in 
Abwurfwinkel, Abwurfgeschwindigkeit oder Abwurfhöhe können die Flugbahn und damit den Wurferfolg maßgeblich beeinflussen. Für eine methodisch kontrollierbare 
Untersuchung bietet sich der Freiwurf an. Er wird aus einer definierten Position ausgeführt, unter weitgehend konstanten Randbedingungen und ohne unmittelbaren 
defensiven Druck. Damit reduziert sich die Variabilität externer Störeinflüsse, was den Freiwurf als standardisierte Basissituation für Modellierung und Vergleich 
besonders geeignet macht.

Auch aus sportpraktischer Perspektive ist der Freiwurf relevant. Empirische Analysen aus dem US-amerikanischen College-Basketball zeigen, dass rund 20\,\% aller erzielten 
Punkte durch Freiwürfe resultierten. Ergänzend wird berichtet, dass bei gleichstarken Teams in 80\,\% der Fälle jene Mannschaft gewinnt, die über die höhere Freiwurfquote 
verfügte ~\cite{kozar1994}. Der Freiwurf ist damit nicht nur ein grundlegendes technisches Element, sondern auch ein leistungsentscheidender Faktor. 
Folglich eignet er sich als Untersuchungseinheit dieser Arbeit, um Wurfdaten unter kontrollierten Bedingungen systematisch auszuwerten.

Vor diesem Hintergrund eröffnet Maschinelles Lernen einen technologischen Rahmen, um aus Wurfdaten Trajektorien vorherzusagen und Rückschlüsse auf zugrunde liegende 
Abwurfparameter zu ziehen. Diese Arbeit grenzt sich dabei bewusst von der Fragestellung ab, eine allgemeingültige „optimale“ Flugbahn herzuleiten, 
die Sportlerinnen und Sportler idealerweise reproduzieren sollten. Im Mittelpunkt steht stattdessen ein personenbezogenes Anwendungsszenario im Sinne 
einer personalisierten Trainingsassistenz: Auf Basis der Daten eines spezifischen Sportlers bzw. einer spezifischen Sportlerin sollen Modelle individuelle Muster 
identifizieren und daraus sportartspezifisch interpretierbare Hinweise für die Verbesserung ableiten. Der praktische Mehrwert liegt damit nicht in einer normativen 
Vorgabe, sondern in diagnostischer Unterstützung. Aus den Analysen sollen konkrete, sportartspezifisch interpretierbare Hinweise entstehen.

\section{Zielsetzung und Forschungsfragen}

\section{Aufbau der Arbeit}