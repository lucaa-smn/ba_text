\chapter{Theoretische Grundlagen}
\label{cha:grundlagen}

Dieses Kapitel bündelt die theoretischen Grundlagen, die zum Verständnis der in Kapitel 5 beschriebenen Modellierung und Implementierung erforderlich sind. 
Ausgangspunkt ist das ballistische Modell nach Haußer \& Luchko, da es die physikalische Referenzdynamik vorgibt und zugleich die Basis für die 
Generierung des synthetischen Datensatzes bildet. Darauf aufbauend werden mit der Encoder-Decoder-Architektur für Zeitreihen und den 
Long Short-Term Memory Networks zunächst die zentralen Bausteine datengetriebener Sequenzmodelle eingeführt. 
Anschließend werden die Prinzipien des physikinformierten Lernens und physikinformierter Neuraler Netze hergeleitet, um die Integration des 
ballistischen Vorwissens in lernbasierte Modelle konzeptionell zu fundieren. Den Abschluss bildet die amortisierte Inferenz, da sie die methodische 
Klammer für die effiziente Parameterschätzung darstellt.

\section{Ballistisches Modell nach Haußer \& Luchko}
Im Folgenden wird das ballistische Freiwurfmodell nach Haußer und Luchko eingeführt, das in dieser Arbeit als physikalische Referenz sowie als Grundlage zur Generierung 
synthetischer 2D-Flugbahnen dient. Haußer und Luchko untersuchen im Abschnitt \enquote{Erstes Modell: Der beste Abwurfwinkel} das Problem, wie ein Spieler gegebener 
Körpergröße den Ball so werfen sollte, dass Abweichungen im Abwurfwinkel möglichst tolerant gegenüber dem Trefferausgang sind.~\cite{hausser2019}

Zur analytischen Behandlung formulieren Haußer und Luchko ein vereinfachtes Modell und treffen hierzu restriktive Annahmen: Es werden ausschließlich \enquote{nearly 
nothing but net}-Treffer betrachtet (direkter Netzeintritt oder Kontakt mit der Ringhinterkante mit zusätzlicher Bedingung, dass der Ballmittelpunkt beim Kontakt auf 
oder unter Ringhöhe liegt). Weiterhin werden Luftwiderstand und Ballspin vernachlässigt, seitliche Abweichungen ausgeschlossen (Reduktion auf eine vertikale Wurfbene) 
und in diesem ersten Modell die Abwurfgeschwindigkeit als fehlerfrei angenommen. Schließlich wird ein \enquote{Idealwurf} definiert, indem zu jedem Abwurfwinkel 
zunächst eine Abwurfgeschwindigkeit so gewählt wird, dass die Flugbahn durch das Zentrum des Rings verläuft; anschließend werden bei festem $v_0$ Winkelabweichungen 
um diesen Idealwurf betrachtet.~\cite{hausser2019}

Die Bewegung wird durch den Ortsvektor $s(t)=(x(t),y(t))^\top\in\mathbb{R}^2$ des Ballmittelpunkts und die Geschwindigkeit $v(t)=\dot{s}(t)=(v_x(t),v_y(t))^\top$ 
beschrieben.~\cite{hausser2019} Die Koordinatenwahl erfolgt mit Ursprung im Abwurfpunkt ($x(0)=0$, $y(0)=0$); der Korb befindet sich in horizontaler Entfernung $l$ und 
in vertikaler Höhe $h$ relativ zum Abwurfpunkt. Dabei wird $h$ als Differenz zwischen Korbhöhe $h_\text{Korb}$ und Abwurfhöhe modelliert, wobei Haußer und Luchko die 
Abwurfhöhe als $\tfrac{5}{4}h_S$ (mit Spielergröße $h_S$) ansetzen, also $h=h_\text{Korb}-\tfrac{5}{4}h_S$. Zusätzlich gehen Ringradius $R_r$, Ballradius $R_b$ sowie 
die Erdbeschleunigung $g$ als Parameter ein.~\cite{hausser2019}

Unter Vernachlässigung dissipativer Kräfte wirkt auf den Ball lediglich die konstante Gravitationsbeschleunigung in negativer $y$-Richtung. Damit ergibt sich das 
Anfangswertproblem
\begin{align}
\dot{x}(t) &= v_x(t), &
\dot{y}(t) &= v_y(t), &
\dot{v}_x(t) &= 0, &
\dot{v}_y(t) &= -g,
\end{align}
mit Anfangsbedingungen
\begin{align}
x(0) &= 0, &
y(0) &= 0, &
v_x(0) &= v_x^0=v^0\cos\theta^0, &
v_y(0) &= v_y^0=v^0\sin\theta^0,
\end{align}
wobei $\theta^0$ den Abwurfwinkel und $v^0$ die skalare Abwurfgeschwindigkeit bezeichnen. Durch Integration folgen die geschlossenen Lösungen
\begin{align}
x(t) &= v_x^0\,t = v^0\cos(\theta^0)\,t,\\
y(t) &= v_y^0\,t - \frac{1}{2}gt^2 = v^0\sin(\theta^0)\,t - \frac{1}{2}gt^2.
\end{align}

Für die spätere Winkelanalyse wird der Zeitpunkt $T$ betrachtet, zu dem der Ball die Korbhöhe $h$ erreicht ($y(T)=h$):
\begin{equation}
T=\frac{v_y^0+\sqrt{(v_y^0)^2-2gh}}{g},
\end{equation}
sofern $(v_y^0)^2\ge 2gh$. In Übereinstimmung mit der Definition des Idealwurfs wählen Haußer und Luchko zu einem gegebenen $\theta^0$ zunächst $v^0$ so, dass die 
Flugbahn durch das Ringzentrum $(l,h)$ verläuft; daraus ergibt sich eine explizite Zuordnung $v^0=v^0(\theta^0)$.~\cite{hausser2019}

Schließlich werden Treffer im Sinne der \enquote{nearly nothing but net}-Annahme über geometrische Nebenbedingungen relativ zu Ring- und Ballradius 
operationalisiert (u.\,a.\ Lage des Ballmittelpunkts beim Erreichen der Korbhöhe innerhalb der Ringöffnung sowie Kollisionsfreiheit gegenüber der vorderen Ringkante 
entlang der Bahn). Diese Bedingungen bilden eine wesentliche Grundlage für die spätere Label-Definition (Kurzwurf/Treffer/Langwurf) und für physikinformierte 
Nebenbedingungen in lernbasierten Modellen.~\cite{hausser2019}


\section{Rekurrente Neurale Netze (RNNs) und LSTMs}

\section{Encoder–Decoder-Architektur}

\section{Physikinformiertes Lernen (PINNs)}

\section{Amortisierte Inferenz}