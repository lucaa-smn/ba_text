\chapter{Theoretische Grundlagen}
\label{cha:grundlagen}

Dieses Kapitel bündelt die theoretischen Grundlagen, die zum Verständnis der in Kapitel 5 beschriebenen Modellierung und Implementierung erforderlich sind. 
Ausgangspunkt ist das ballistische Modell nach Haußer \& Luchko, da es die physikalische Referenzdynamik vorgibt und zugleich die Basis für die 
Generierung des synthetischen Datensatzes bildet. Darauf aufbauend werden mit der Encoder-Decoder-Architektur für Zeitreihen und den 
Long Short-Term Memory Networks zunächst die zentralen Bausteine datengetriebener Sequenzmodelle eingeführt. 
Anschließend werden die Prinzipien des physikinformierten Lernens und physikinformierter Neuraler Netze hergeleitet, um die Integration des 
ballistischen Vorwissens in lernbasierte Modelle konzeptionell zu fundieren. Den Abschluss bildet die amortisierte Inferenz, da sie die methodische 
Klammer für die effiziente Parameterschätzung darstellt.

\section{Ballistisches Modell nach Haußer \& Luchko}
Im Folgenden wird zunächst das ballistische Modell nach Haußer \& Luchko~\cite{hausser2019} eingeführt, das als physikalischer Ausgangspunkt und Referenz für die in dieser Arbeit 
untersuchten Wurftrajektorien dient. Sie stellen sich in ihrer Arbeit unter anderem die Aufgabe der Modellierung des Freiwurfs beim Basketball.
Im Rahmen des Kapitels \enquote{Erstes Modell: Der beste Abwurfwinkel} bearbeiten Sie das folgende Anwendungsproblem: \enquote{Wie sollte ein Basketballspieler 
einer bestimmten Größe beim Freiwurf den Ball werfen, damit eine größtmögliche Abweichung beim Abwurf immer noch zu einem Treffer führt?}~\cite{hausser2019} Dabei 
erstellen sie zur Beantwortung ihrer Fragestellung ein ballistisches Modell, das die wesentlichen physikalischen Einflüsse auf die Flugbahn des Balls berücksichtigt. 
Zur Analyse des Problems stellen Sie vereinfachte Annahmen auf, die im Folgenden kurz zusammengefasst werden:
\begin{itemize}
    \item \textbf{1.} \enquote{Nur „Nearly nothing but net“ -Würfe. Damit ist folgendes gemeint: Wir betrachten als mögliche Treffer nur (a) Würfe, bei denen der Ball direkt ins Korbnetz geht (also ohne Berührung des Rings oder des Backboards) oder (b) Würfe, bei denen der Ball die Hinterkante des Ringes trifft und dann direkt ins Netz geht.}~\cite{hausser2019}
    \item \textbf{2.} \enquote{Vernachlässigung des Luftwiderstandes.}~\cite{hausser2019}
    \item \textbf{3.} \enquote{Vernachlässigung des Spins. Der Spin des Balls (also das Drehen um die eigene Achse) wird besonders wichtig, wenn der Ball vom Ring oder vom Backboard abprallt, bevor er in den Korb geht. Wegen Annahme 1 können wir auch den Spin vernachlässigen.}~\cite{hausser2019}
    \item \textbf{4.} \enquote{Kein seitlicher Fehler der Flugbahn. Wir gehen davon aus, dass ein guter Werfer ziemlich genau geradeaus werfen kann.}~\cite{hausser2019}
    \item \textbf{5.} \enquote{Kein Fehler in der Abwurfgeschwindigkeit.}~\cite{hausser2019}
    \item \textbf{6.} \enquote{Der ideale Wurf geht durch das Zentrum des Rings.}~\cite{hausser2019}
\end{itemize} 

Die relavaten physikalischen Variablen, die zur Beschreibung des Zustandes herangezogen werden, sind der Ort \(s(t)\) und die Geschwindigkeit \(v(t)\) des Balls 
zur Zeit \(t\). \enquote{Wegen Annahme~4 ist die Bewegung zweidimensional, also
\( s(t) \in \mathbb{R}^2\)\index{R2@\ensuremath{\mathbb{R}^2}}
und
\( v(t) \in \mathbb{R}^2\)\index{R2@\ensuremath{\mathbb{R}^2}},
und aufgrund von Annahme~3 gibt es keine weiteren Bewegungsfreiheitsgrade des Balls.}~\cite{hausser2019}
Die Wurfbahn des Balls ergibt sich aus den Newtonschen Bewegungsgleichungen. \(s(t)\) beschreibt den Ort des Ballmittelpunkts zum Zeitpunkt \(t\) und gibt damit die
Trajektorie des Balls an. Das Modell nimmt an, dass der Ball zum Zeitpunkt \(t=0\) mit einem Abwurfwinkel \(\theta\) und einer (skalaren) Abwurfgeschwindigkeit \(v_0\)
abgeworfen wird. Die Wurfparabel wird durch die horizontale Komponente \(x(t)\) und die vertikale Komponente \(y(t)\) beschrieben, also \(s(t) = (x(t), y(t))\).
Nach Annahme 2 bewegt sich der Ball in horizontaler Richtung mit konstanter Geschwindigkeit, d.h. es gilt: \(v_x = v_x^0\). In vertikaler Richtung wirkt die 
Gravitationskraft \(g\) nach unten auf den Ball: \(\dot{v}_y = -g\). Hier bezeichet der Punkt über einer Variablen die zeitliche Ableitung \(d/dt\) nach der Zeit \(t\).
Einfach Integration ergbit dann \(v_y = v_y^0-gt\). Wegen \(v(t) = \dot{s}(t)\) ergbit sich durch eine weitere Integration und einsetzen der Anfangsbedingungen 
\(x(0) = 0\), \(y(0) = 0\) als Lösung die Wurfparabel:

\begin{equation}
    x(t) = v_x^0(t)= v^0 \cos(\theta) t, t >= 0
\end{equation}
\begin{equation}
    y(t) = v_y^0(t) - \frac{1}{2}gt^2 = v^0 \sin(\theta^0) t - \frac{1}{2}gt^2, t >= 0
\end{equation}



\section{Encoder–Decoder-Architektur für Zeitreihen}

\section{LSTM}

\section{Physikinformiertes Lernen \& PINNs}

\section{Amortisierte Inferenz}