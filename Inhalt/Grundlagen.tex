\chapter{Theoretische Grundlagen}
\label{cha:grundlagen}

\section{Ballistisches Modell nach Haußer \& Luchko}

Für die Modellierung des Freiwurfs sind vor allem die geometrischen Maße des Spielfeldes, Korbs und Balls ausschlaggebend. Wie in Abbildung~\ref{fig:nba_court} zu erkennen misst das Spielfeld nach den offiziellen NBA-Regeln 94\,ft in der Länge und 50\,ft in der Breite. Die Korbanlagen sind jeweils mittig an den Grundlinien positioniert, wobei die Vorderfläche des Backboards 4\,ft von der Grundlinie entfernt ist. Das Brett selbst ist rechteckig und misst 6\,ft in der Breite und 3.5\,ft in der Höhe. Es ist in der Regel transparent ausgeführt. Der Ring des Korbs befindet sich 10\,ft (ca. 3.05\,m) über dem Spielfeldboden. Er ist an das Backboard montiert. Der horizontale Abstand zwischen dem Backboard und der Freiwurflinie beträgt 15\,in. Die Normen für den Ball der Männer liegen bei einen Umfang zwischen 75 und 78\,cm und einem Gewicht zwischen 600 und 650\,g. Sein Radius ist als Intervall angegeben und die Oberfläche ist aus Leder und Kunststoff~\cite{nba_rules}.

\begin{figure}[htbp]
    \centering
    \includegraphics[width=0.9\textwidth]{Bilder/nba_court.png}
    \caption{Offizielle Maße eines NBA-Spielfelds. \cite{nba_rules}}
    \label{fig:nba_court}
\end{figure}

Für eine präzise Modellierung ist es notwendig, sowohl die grundlegenden Kraftwirkungen als auch die spezifischen Parameter der Wurfbewegung systematisch zu betrachten. Während die Gravitation eine konstante Einflussgröße darstellt, variieren andere Parameter – etwa Spin oder Abwurfhöhe – in Abhängigkeit von der individuellen Technik des Spielers. Diese Variabilität macht es erforderlich, die relevanten physikalischen Größen genau zu definieren und ihre Auswirkungen auf die Flugbahn des Balls zu analysieren. Im Folgenden werden die wichtigsten Kräfte und Parameter erläutert, die für die Modellierung des Freiwurfs von Bedeutung sind.

\section{Encoder–Decoder-Architektur für Zeitreihen}

\section{LSTM}

\section{Physikinformiertes Lernen \& PINNs}

\section{Amortisierte Inferenz}